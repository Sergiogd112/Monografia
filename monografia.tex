%=-=-=-=-=-=-=-=-=-=-=-=-=-=-=-=-=-=-=-=-=-=-=-=-=-=-=-=
% INTERNATIONA BACCALEAREATE INTERNAL ASSESSMENT TEMPLATE
%
%   Name        ::  ia_template.tex
%	Author		::  Mark Olson
%	Created		::  2017-06-05
%	Updated		::  2018-09-28 14:50
%	Version		::  0.1
%	Github		::  markolsonse
%
%=-=-=-=-=-=-=-=-=-=-=-=-=-=-=-=-=-=-=-=-=-=-=-=-=-=-=-=

%-=-=-=-=-=-=-=-=-=-=-=-=-=-=-=-=-=-=-=-=-=-=-=-=
%	DOCUMENT CLASS "TEMPLATE"
%-=-=-=-=-=-=-=-=-=-=-=-=-=-=-=-=-=-=-=-=-=-=-=-=
\documentclass[a4paper,12pt]{article}
\usepackage{ibcat5ia}
\usepackage{opencolor}
\usepackage[utf8]{inputenc}


%-=-=-=-=-=-=-=-=-=-=-=-=-=-=-=-=-=-=-=-=-=-=-=-=
%	DOCUMENT INFORMATION - COMPLETE THIS SECTION
%-=-=-=-=-=-=-=-=-=-=-=-=-=-=-=-=-=-=-=-=-=-=-=-=

\title{Interpolation error analisis}
\author{Sergio Gómez Damas}
\date{Version: 0.1}

\begin{document}
\maketitle
\newpage

\tableofcontents
\newpage

%-=-=-=-=-=-=-=-=-=-=-=-=-=-=-=-=-=-=-=-=-=-=-=-=
%	YOUR ABSTRACT GOES HERE - SELL YOU PAPER
%-=-=-=-=-=-=-=-=-=-=-=-=-=-=-=-=-=-=-=-=-=-=-=-=
\section{Abstract}
  In this project, we analyze different types of interpolation based on the error when upscaling an image. We will also compare it with new techniques such as ProSR. By doing so, we will be able to identify which one is the best for each occasion.

%-=-=-=-=-=-=-=-=-=-=-=-=-=-=-=-=-=-=-=-=-=-=-=-=
%
%	SECTION: Introduction
%
%-=-=-=-=-=-=-=-=-=-=-=-=-=-=-=-=-=-=-=-=-=-=-=-=

\section{Introduction}\label{sec:introduction}
Nowadays, thanks to the high resolution of the sensors and the computers being able to do complex simulations, more precise measurements and analysis can be made. Some times there is not enough resolution to analyze the details. Due to this, there have been different techniques to overcome this problem. Those methods are interpolations. In this project, they will be analyzed and classified based on their error.
\subsection{Objectives}
The objective of this project is to analyze the different methods of interpolation. Also, compare them against new techniques like ProSR and try to design one based on Fourier transforms.
\subsection{Motivation}



%-=-=-=-=-=-=-=-=-=-=-=-=-=-=-=-=-=-=-=-=-=-=-=-=
%
%	SECTION: Methodology
%
%-=-=-=-=-=-=-=-=-=-=-=-=-=-=-=-=-=-=-=-=-=-=-=-=

\section{Methodology}\label{sec:methodology}
Each one will be tested on different types of images. Those images will be first downscaled to one-quarter of the size and then upscaled using each method. The image generated will be subtracted to the original and the result will be displayed as a histogram.
One of the images will be a representation of Mandelbrot's set on the complex plain. The other images will be extracted from creative common archives.
\subsection{Interpolations}\label{subsec:interpolaions}
Interpolation, in mathematics, the determination or estimation of the value of $f(x)$, or a function of x, from certain known values of the function. If $x0 < \dots< xn$ and $y0 = f(x0),\dots, yn = f(xn)$ are known, and if $x0 < x < xn$, then the estimated value of $f(x)$ is said to be an interpolation. If $x < x0$ or $x > xn$, the estimated value of $f(x)$ is said to be an extrapolation.\cite{britanica:interpolationdef}


\subsubsection{Nearest neighbor}
Let a function $f$ defined on a discrete subset $D$ of a metric space $E$ (it can be the real plane, but it does not really matter). The nearest-neighbor interpolation of $f$ is another function $g$ that's defined everywhere, and whose value on any point $x\in{E}$ is exactly the value of f on the nearest point $y\in{D}$ to $x$, where it exists, and it does not really matter what you chose when the nearest point is non-unique.

On the open-dense subset where $g$ is defined, it is piecewise-constant, you can easily prove that in a small enough ball around a point that has a nearest neighbor, all the points in that ball have the same nearest neighbor. So the derivative in those areas is $0$.

It remains to be seen what happens on the boundaries between those regions. Remember that we haven't really defined g there. Now, if the point $x$ has several nearest neighbors, it may happen that in all these neighbors f has the same value. In that case you can extend $g$ by continuity, and it will still be constant.

If on the other case f takes different values at the nearest neighbors of $x$, then there is no way to define $g(x)$ for the function to be continuous, much less differentiable, because in every ball around $x$ you will have points where $g$ assumes different values.
\subsubsection{Bilinear}

Suppose that we want to find the value of the unknown function ''f'' at the point (''x'', ''y''). It is assumed that we know the value of ''f'' at the four points ''Q''<sub>11</sub> = (''x''<sub>1</sub>,&nbsp;''y''<sub>1</sub>), ''Q''<sub>12</sub> = (''x''<sub>1</sub>,&nbsp;''y''<sub>2</sub>), ''Q''<sub>21</sub> = (''x''<sub>2</sub>,&nbsp;''y''<sub>1</sub>), and ''Q''<sub>22</sub> = (''x''<sub>2</sub>,&nbsp;''y''<sub>2</sub>).

We first do linear interpolation in the ''x''-direction. This yields
\begin{align}
f(x, y_1) &\approx \frac{x_2-x}{x_2-x_1} f(Q_{11}) + \frac{x-x_1}{x_2-x_1} f(Q_{21}), \\
f(x, y_2) &\approx \frac{x_2-x}{x_2-x_1} f(Q_{12}) + \frac{x-x_1}{x_2-x_1} f(Q_{22}).
\end{align}
We proceed by interpolating in the ''y''-direction to obtain the desired estimate:
\begin{align}
f(x,y) &\approx \frac{y_2-y}{y_2-y_1} f(x, y_1) + \frac{y-y_1}{y_2-y_1} f(x, y_2) \\
&= \frac{y_2-y}{y_2-y_1} \left ( \frac{x_2-x}{x_2-x_1} f(Q_{11}) + \frac{x-x_1}{x_2-x_1} f(Q_{21}) \right ) + \frac{y-y_1}{y_2-y_1} \left ( \frac{x_2-x}{x_2-x_1} f(Q_{12}) + \frac{x-x_1}{x_2-x_1} f(Q_{22}) \right ) \\
&= \frac{1}{(x_2-x_1)(y_2-y_1)} \big( f(Q_{11})(x_2-x)(y_2-y) + f(Q_{21})(x-x_1)(y_2-y)+  f(Q_{12})(x_2-x)(y-y_1) + f(Q_{22})(x-x_1)(y-y_1) \big)\\
&=\frac{1}{(x_2-x_1)(y_2-y_1)}  \begin{bmatrix} x_2-x & x-x_1 \end{bmatrix} \begin{bmatrix} f(Q_{11}) & f(Q_{12}) \\ f(Q_{21})& f(Q_{22}) \end{bmatrix} \begin{bmatrix}
y_2-y \\ y-y_1 \end{bmatrix}.
\end{align}
Note that we will arrive at the same result if the interpolation is done first along the $y$ direction and then along the $x$ direction.\cite{Cambrige:Numerical_Recepies}
\subsubsection{Bicubic}
In mathematics, bicubic interpolation is an extension of cubic interpolation for interpolating data points on a two-dimensional regular grid. The interpolated surface is smoother than corresponding surfaces obtained by bilinear interpolation or nearest-neighbor interpolation. Bicubic interpolation can be accomplished using either Lagrange polynomials, cubic splines, or cubic convolution algorithm. In this project the algorithm used is splines.

The algorithm consists in a third-order polynomial $q(x)$ for which
$q(x_1)=y_1</math>$
$q(x_2)=y_2</math>$
$q'(x_1)=k_1</math>$
$q'(x_2)=k_2</math>$

can be written in the symmetrical form
$q(x) =\big(1-t(x)\big)\,y_1 + t(x)\,y_2+ t(x)\big(1-t(x)\big) \Big(\big(1-t(x)\big)\,a + t(x)\,b\Big)$
where
$>t(x)=\frac{x-x_1}{x_2-x_1},$
$a= k_1 (x_2 - x_1)-(y_2 - y_1)$
$b=-k_2 (x_2 - x_1)+(y_2 - y_1)$
As
$q'= \frac{d q}{d x} = \frac{d q}{d t} \frac{d t}{d x} = \frac{d q}{d t} \frac{1}{x_2-x_1}$
one gets that:
$q'=\frac {y_2-y_1}{x_2-x_1} +(1-2t) \frac {a(1-t) + b t}{x_2-x_1}  +t(1-t) \frac {b-a}{x_2-x_1},$
$q''=2\frac {b-2a+(a-b)3t}{{(x_2-x_1)}^2}$

Setting {{math|''x'' {{=}} ''x''<sub>1</sub>}} and {{math|''x'' {{=}} ''x''<sub>2</sub>}} respectively in equations ({{EquationNote|5}}) and ({{EquationNote|6}}) one gets from ({{EquationNote|2}}) that indeed first derivatives {{math|''q′''(''x''<sub>1</sub>) {{=}} ''k''<sub>1</sub>}} and {{math|''q′''(''x''<sub>2</sub>) {{=}} ''k''<sub>2</sub>}} and also second derivatives

{{NumBlk|:|<math>q''(x_1)=2\frac {b-2a}{{(x_2-x_1)}^2}</math>|{{EquationRef|7}}}}
{{NumBlk|:|<math>q''(x_2)=2\frac {a-2b}{{(x_2-x_1)}^2}</math>|{{EquationRef|8}}}}

If now {{math|(''x<sub>i</sub>'', ''y<sub>i</sub>''), ''i'' {{=}} 0, 1, ..., ''n''}} are {{math|''n'' + 1}} points and

{{NumBlk|:|<math>q_i = (1-t)\,y_{i-1} + t\,y_i + t(1-t)\big((1-t)\,a_i + t\,b_i\big)</math>|{{EquationRef|9}}}}

where ''i'' = 1, 2, ..., ''n'' and <math>t=\tfrac{x-x_{i-1}}{x_{i}-x_{i-1}}</math> are ''n'' third degree polynomials interpolating {{mvar|y}} in the interval {{math|''x''<sub>''i''−1</sub> ≤ ''x'' ≤ ''x<sub>i</sub>''}} for ''i'' = 1, ..., ''n'' such that {{math|''q′<sub>i</sub>'' (''x<sub>i</sub>'') {{=}} ''q′''<sub>''i''+1</sub>(''x<sub>i</sub>'')}} for ''i'' = 1, ..., ''n''−1  then the ''n'' polynomials together define a differentiable function in the interval {{math|''x''<sub>0</sub> ≤ ''x'' ≤ ''x<sub>n</sub>''}} and

{{NumBlk|:|<math>a_i=k_{i-1}(x_i-x_{i-1})-(y_i - y_{i-1})</math>|{{EquationRef|10}}}}
{{NumBlk|:|<math>b_i=-k_i(x_i-x_{i-1})+(y_i - y_{i-1})</math>|{{EquationRef|11}}}}
for ''i'' = 1, ..., ''n'' where
{{NumBlk|:|<math>k_0=q_1'(x_0)</math>|{{EquationRef|12}}}}
{{NumBlk|:|<math>k_i=q_i'(x_i)=q_{i+1}'(x_i) \qquad i=1,\dotsc ,n-1</math>|{{EquationRef|13}}}}
{{NumBlk|:|<math>k_n=q_n'(x_n)</math>|{{EquationRef|14}}}}

If the sequence {{math|''k''<sub>0</sub>, ''k''<sub>1</sub>, ..., ''k<sub>n</sub>''}} is such that, in addition, {{math|''q′′<sub>i</sub>''(''x<sub>i</sub>'') {{=}} ''q′′''<sub>''i''+1</sub>(''x<sub>i</sub>'')}} holds for ''i'' = 1, ..., ''n''-1, then the resulting function will even have a continuous second derivative.

From ({{EquationNote|7}}), ({{EquationNote|8}}), ({{EquationNote|10}}) and ({{EquationNote|11}}) follows that this is the case if and only if

${{NumBlk|:|<math>\frac {k_{i-1}}{x_i-x_{i-1}} + \left(\frac {1}{x_i-x_{i-1}}+ \frac {1}{{x_{i+1}-x_i}}\right) 2k_i+ \frac {k_{i+1}}{{x_{i+1}-x_i}} =
   3 \left(\frac {y_i - y_{i-1}}{{(x_i-x_{i-1})}^2}+\frac {y_{i+1} - y_i}{{(x_{i+1}-x_i)}^2}\right)</math>|{{EquationRef|15}}}}$

for ''i'' = 1, ..., ''n''-1. The relations ({{EquationNote|15}}) are {{math|''n'' − 1}} linear equations for the {{math|''n'' + 1}} values {{math|''k''<sub>0</sub>, ''k''<sub>1</sub>, ..., ''k<sub>n</sub>''}}.

For the elastic rulers being the model for the spline interpolation one has that to the left of the left-most "knot" and to the right of the right-most "knot" the ruler can move freely and will therefore take the form of a straight line with {{math|''q′′'' {{=}} 0}}. As {{mvar|q′′}} should be a continuous function of {{mvar|x}} one gets that for "Natural Splines" one in addition to the {{math|''n'' − 1}} linear equations ({{EquationNote|15}}) should have that
:<math>q''_1(x_0) =2 \frac {3(y_1 - y_0)-(k_1+2k_0)(x_1-x_0)}{{(x_1-x_0)}^2}=0,</math>
:<math>q''_n(x_n) =-2 \frac {3(y_n - y_{n-1})-(2k_n+k_{n-1})(x_n-x_{n-1})}{{(x_n-x_{n-1})}^2}=0,</math>
i.e. that
\begin{equation}
    \frac{2}{x_1-x_0} k_0 +\frac{1}{x_1-x_0}k_1 = 3
    \frac{y_1-y_0}{(x_1-x_0)^2}
    \frac{1}{x_n-x_{n-1}}k_{n-1} +\frac{2}{x_n-x_{n-1}}k_n = 3 \frac{y_n-y_{n-1}}{(x_n-x_{n-1})^2}

\end{equation}

Eventually, ({{EquationNote|15}}) together with ({{EquationNote|16}}) and ({{EquationNote|17}}) constitute {{math|''n'' + 1}} linear equations that uniquely define the {{math|''n'' + 1}} parameters {{math|''k''<sub>0</sub>, ''k''<sub>1</sub>, ..., ''k<sub>n</sub>''}}.

There exist other end conditions: "Clamped spline", that specifies the slope at the ends of the spline, and the popular "not-a-knot spline", that requires that the third derivative is also continuous at the $x_1$ and $x_{N-1}$ points.
For the "not-a-knot" spline, the additional equations will read:

$q'''_{n-1}(x_{n-1}) =q'''_n(x_{n-1}) \Rightarrow \frac{1}{\Delta x_{n-1}^2} k_{n-2} + \left( \frac{1}{\Delta x_{n-1}^2} - \frac{1}{\Delta x_n^2} \right) k_{n-1} - \frac{1}{\Delta x_n^2} k_n = 2\left( \frac{\Delta y_{n-1} }{\Delta x_{n-1}^3 }- \frac{ \Delta y_n}{ \Delta x_n^3 } \right)$

where $ \Delta x_i = x_i - x_{i-1}, \Delta y_i = y_i - y_{i-1}$.


\subsubsection{Polinomial}

\subsubsection{Splines}

\subsubsection{Multivariate}

\subsubsection{Area}

\subsubsection{LANCZOS4}

\newpage

%-=-=-=-=-=-=-=-=-=-=-=-=-=-=-=-=-=-=-=-=-=-=-=-=
%	YOUR IA CONTENT GOES HERE
%-=-=-=-=-=-=-=-=-=-=-=-=-=-=-=-=-=-=-=-=-=-=-=-=

\printbibliography

\end{document}
